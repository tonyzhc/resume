% !TEX TS-program = xelatex
% !TEX encoding = UTF-8 Unicode
% !Mode:: "TeX:UTF-8"

\documentclass{resume}
\usepackage{zh_CN-Adobefonts_external} % Simplified Chinese Support using external fonts (./fonts/zh_CN-Adobe/)
% \usepackage{NotoSansSC_external}
% \usepackage{NotoSerifCJKsc_external}
% \usepackage{zh_CN-Adobefonts_internal} % Simplified Chinese Support using system fonts
\usepackage{linespacing_fix} % disable extra space before next section
\usepackage{cite}

\begin{document}
\pagenumbering{gobble} % suppress displaying page number

\name{张翰驰}

\basicInfo{
  \email{tonyzhanghc@hotmail.com} \textperiodcentered\ 
  \phone{(+1) 734-882-7007} \textperiodcentered\ 
  \linkedin[tonyhanchizhang]{https://www.linkedin.com/in/tonyhanchizhang} \textperiodcentered\
  \github[tonyzhc]{https://github.com/tonyzhc}} 
 
\section{\faGraduationCap\  教育背景}
\datedsubsection{\textbf{密西根大学安娜堡分校} - 安娜堡,美国}{2017 / 09 -- 2020 / 05}
\textit{在读本科学士}\ 计算机科学/数据科学 \hfill GPA 3.84/4.00

\section{\faUsers\ 实习经历}
\datedsubsection{\textbf{Perch Research} - 安娜堡,美国}{2018 / 12 -- 至今}
\role{前端开发}{HTML / CSS / javascript / React}
\vspace{-2mm}
\begin{onehalfspacing}
\begin{itemize}
	\item \href{http://umich.edu/~perchres/}{\underline{Perch}}是一个学生初创的在线平台,致力于帮助本科生找到他们感兴趣的研究项目并提供相关训练
	\item 使用useEffect / useState等React Hooks API重构前端模组,去除了繁琐的类及函数 声明,统一使用函数声明模组,提高了代码可读性和重复使用率
	\item 利用库emotion / styled component声明inline css styling,省略了不必要的类声明并简化了文件结构
\end{itemize}
\end{onehalfspacing}

\datedsubsection{\textbf{北京小米科技} - 北京,中国}{2018 / 06 -- 2018 / 08}
\role{软件开发实习生}{Python / Tensorflow / Java / Android}
\vspace{-2mm}
\begin{onehalfspacing}
\begin{itemize}
	\item 优化了广告推荐系统的神经网络中的超参数,并通过阅读论文尝试了其它不同种类的神经网络(如wide and deep),将该网络的AUC提高了2\%
	\item 开发了一个向用户提供日常MIUI App使用数据的安卓程序demo,使用了安卓内置的用户数据检测包,同时使用了包括OkHttp和MPAndroidChart在内的开源库提供更好的图形化用户体验
	\item 使用Python库Flask搭建了模拟服务器及提供API以测试与安卓程序的网络连接并传输模拟数据
\end{itemize}
\end{onehalfspacing}

% Reference Test
%\datedsubsection{\textbf{Paper Title\cite{zaharia2012resilient}}}{May. 2015}
%An xxx optimized for xxx\cite{verma2015large}
%\begin{itemize}
%  \item main contribution
%\end{itemize}

\section{\faCogs\ 技能}
% increase linespacing [parsep=0.5ex]
\begin{itemize}[parsep=0.5ex]
  \item 编程语言: C++, C, Python, Java, HTML, CSS,  javascript, SQL, \LaTeX, Swift (初级)
  \item 工具: Git, Bash, Tensorflow, React, Node.js, jQuery
  \item 其它: 英语 - 熟练(6年海外学习经历)
\end{itemize}

\section{\faTrophy\ 获奖}
\datedline{\textit{Best Voice Tech Design}, University of Michigan Makeathon}{2019 / 02}
\begin{onehalfspacing}
\textit{参与指导的亚马逊Solution Architect \href{https://www.linkedin.com/in/jpd44/}{\underline{John Dixon}}曾在Medium上发表过关于我们项目的\href{https://medium.com/voice-tech-podcast/alexa-when-is-the-next-bus-620e1aba9474}{\underline{博客}}}
\begin{itemize}
	\item 参与了由亚马逊Alexa赞助的Makeathon活动,在24小时内完成了语音助手平台上的一个公交车时间提醒应用,可以识别用户的问题并给出最近车站的下一班车时间及所需步行时间
	\item 使用库requests和pyQuery编写爬虫抓取学校网站实时公交车时间,并使用Google Distance Matrix API计算当前地点与最近公车站间距离,最后返回相应数据
	\item 利用亚马逊的AWS Lambda服务部署后端应用,使得前端可以直接调用API返回用户所需数据
\end{itemize}
\end{onehalfspacing}

\section{\faWrench\ 项目经历}
\datedsubsection{\textbf{多线程库 / 磁盘调度器 }  \small{C++} }{2019 / 01 -- 2019 / 02}
\begin{onehalfspacing}
\begin{itemize}
	\item 利用Linux系统库实现了支持多个模拟CPU的多线程库,提供了类似于C++标准库的thread、mutex和cv的接口,同时支持处理模拟的CPU间的中断事件并切换至下一个可用的线程
	\item 利用多线程库实现了磁盘调度器程序,在命令行界面读取任意数量的包含磁盘道数的文件,并根据最短寻址时间顺序处理每个request
\end{itemize}
\end{onehalfspacing}

% \datedsubsection{\textbf{Movie Archive Tracker }  \small{C++} }{Sept. 2018 -- Oct. 2018}
% \begin{itemize}
% 	\item Implemented a thread library utilizing linux system infrastructures supporting multiple simulated CPUs
% 	\item Provided full multithreading support for library interfaces such as thread.join() and mutex/condition queue
% 	\item Implemented a disk scheduler program that reads a series of input representing the track number of disk requests that will be executed synchronously by a disk servicer in SSTF order
% \end{itemize}

%% Reference
%\newpage
%\bibliographystyle{IEEETran}
%\bibliography{mycite}
\end{document}
