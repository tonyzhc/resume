\documentclass{resume}
%\usepackage{zh_CN-Adobefonts_external} % Simplified Chinese Support using external fonts (./fonts/zh_CN-Adobe/)
%\usepackage{zh_CN-Adobefonts_internal} % Simplified Chinese Support using system fonts
%\usepackage[top=0.5in, bottom=0.5in]{geometry}
% \addtolength{\oddsidemargin}{-10in}
% \addtolength{\evensidemargin}{-10in}
% \addtolength{\textwidth}{20in}
%\documentstyle[fullpage]{article}

%Shout out to https://github.com/billryan/resume for providing this amazing resume template!
\begin{document}
\pagenumbering{gobble} % suppress displaying page number

\name{Hanchi Zhang}

\basicInfo{
  	\email{tonyzhc@umich.edu} \textperiodcentered\ 
 	\phone{(+1) 734-882-7007} \textperiodcentered\ 
 	\linkedin[tonyhanchizhang]{https://www.linkedin.com/in/tonyhanchizhang} \textperiodcentered\
	% \github[tonyzhc]{https://github.com/tonyzhc} \textperiodcentered\ 
	\homepage[tonyzhc.github.io]{https://tonyzhc.github.io/}
} 

\section{\faGraduationCap\ Education}
\datedsubsection{\textbf{University of Michigan}, Ann Arbor, MI}{Expected May 2021}
\textit{B.S.E.} in Computer Science \& Data Science \hfill GPA 3.75/4.00  \\
% \textbf{Relevant coursework}: EECS 482 (Intro to Operating System) / EECS 491 (Intro to Distributed System) / EECS 381 (Object Oriented and Advanced Programming) / EECS 484 (Database Management System)
\textbf{Relevant coursework}: Intro to Operating System / Intro to Distributed System / Object Oriented and Advanced Programming / Database Management System
%Relevant Courses: Intro to Operating System / Intro to Computer Security / Database System Management

\section{\faUsers\ Experience}
\datedsubsection{\textbf{Google Cloud Platform} Sunnyvale, CA}{Expected May - Aug. 2020}
\role{Incoming Software Engineer Intern}{C++}
\vspace{-2ex}
\begin{itemize}
	\item Will work under the Spanner team, a globally distributed database
\end{itemize}

\datedsubsection{\textbf{University of Michigan Systems Lab} Ann Arbor, MI}{Apr. 2019 - Present}
\role{Research Assistant, advised by Prof. Michael Cafarella}{Python / pytorch}
\vspace{-2ex}
\begin{itemize}
	\item Utilized Python and pyTorch to implement a video database to optimize user-provided pre-trained machine learning model's querying time of per-frame data by building 1.3 million cascades of 360 simple convolutional neural networks
	\item Achieved an optimization of speeding up user querying time by 7 times with 93\% accuracy comparing to naive solution
	%\item Implemented a video database that improves querying time of the user given model by 7 times with 93\% accuracy by constructing and choosing from 1.3 million cascades composing of 360 smaller classifiers
	%\item Shifted metric from accuracy to precision to eliminate 40\% of false positives produced by original method
	\item Processed 4GB of highly noisy csv files to train and fine-tune multiple LSTM classifiers that achieved 75\% near accuracy %and average absolute error of 6 dollars per 225,000 product
\end{itemize}

% \datedsubsection{\textbf{Chemical Engineering at University of Michigan} Ann Arbor, MI}{Apr. 2019 -- Present}
% \role{Websites Manager}{HTML / CSS / javascript / React}
% \vspace{-2ex}
% \begin{itemize}
% 	\item Refactored legacy pages by converting pure HTML pages to React components and utilized emotion.sh to improve styling
% 	\item Seperated contents from styling by storing contents on local json files and enabled easier modification of website contents
% 	% \item Set up webpack / babel hooks to bundle javascript files and embed them into original pages to achieve automatic updates
% \end{itemize}
% \vspace{-1ex}
% %\textit{Checkout the websites \href{http://umich.edu/~elements/5e/index.html}{\underline{here}} and \href{http://umich.edu/~safeche/}{\underline{here}}}
% \textit{View the websites} \underline{http://umich.edu/\textasciitilde elements/5e/index.html} \textit{and} \underline{http://umich.edu/\textasciitilde safeche/}

\datedsubsection{\textbf{Perch Research} Ann Arbor, MI}{Dec. 2018 -- Present}
\role{Frontend Developer}{HTML / CSS / javascript / React / Redux}
\vspace{-2ex}
\begin{itemize}
	\item Developed platform responsive frontend pages and components utilizing React / Redux and media query function of CSS
	\item Adapted to React Hooks API to restructure components declared as classes to functions with useState / useEffect to avoid complex syntactic characteristics of javaScript class and achieve better code readability and reusability
	\item Utilized Youtube API to build a component that enables user to upload and display introductory video with polling support
	%\item Utilized Redux to keep track of complex state changes in components such as a search page
\end{itemize}

\datedsubsection{\textbf{Xiaomi Technology} Beijing, China}{Jun. 2018 -- Aug. 2018}
\role{Software Development Intern}{Python / Java}
\vspace{-2ex}
\begin{itemize}
	\item Developed a multi-activity Android app demo providing intelligent sketch of MIUI users' daily behaviors e.g. app usage
	\item Set up a simulated backend and API endpoint to serve dummy data to the app using Python and flask and store login info
  	\item Optimized the hyperparameters of a deep neural network of the ad system and increased the AUC by 1\%
 	\item Implemented more complex neural network structures such as RNN/W\&D and tested their performance
\end{itemize}

% Reference Test
%\datedsubsection{\textbf{Paper Title\cite{zaharia2012resilient}}}{May. 2015}
%An xxx optimized for xxx\cite{verma2015large}
%\begin{itemize}
%  \item main contribution
%\end{itemize}

\section{\faWrench\ Projects}
\datedsubsection{\textbf{Distributed Key-Value Store} \small{Go}}{Nov. 2019}
\begin{itemize}
	\item Implemented a sharded key-value store that distributes its keys across servers with a central Paxos-based shard master
	\item Build a Paxos library that allows customized definition of operations such that servers reach agreement on a series of logs
	\item Gracefully dealt with the problem of repeated RPC calls such that server space does not shrink as request number increase
\end{itemize}

\datedsubsection{\textbf{U of M Blue Bus Time Checker } \small{Python / pyQuery / AWS} }{Feb. 2019}
% A Makeathon event project sponsored by Amazon co-developed under 24 hours
\begin{itemize}
	\item Developed the backend of an Alexa skill that informs user the time of the next bus and the time to walk to the closest stop 
	\item Built an API endpoint that uses Google Matrix APIs and scraped data from the blue bus website to provide correct info
	\item Utilized a python library \href{https://github.com/aws/chalice}{\underline{Chalice}} to deploy the backend service as a stateless function to AWS Lambda
	% \item Provided data by writing a simple crawler that utilizes requests / pyQuery / BS4 to access the bus time
\end{itemize}
\vspace{-1.5ex}
% \textit{Our project was published as a \href{https://medium.com/voice-tech-podcast/alexa-when-is-the-next-bus-620e1aba9474}{\underline{Medium blog}} by a participating Amazon Solution Architect \href{https://www.linkedin.com/in/jpd44/}{\underline{John Dixon}}}

% \datedsubsection{\textbf{Multithreaded Library / Disk Scheduler }  \small{C++} }{Jan. 2019 -- Feb. 2019}
% \begin{itemize}
% 	\item Implemented a thread library utilizing linux system infrastructures including context and atomic that supports multithreading across multiple simulated CPU cores including interface for mutex and condition queue
% 	\item Implemented a disk scheduler program that reads a series of input representing the track number of disk requests that will be executed synchronously by a disk servicer
% \end{itemize}

\section{\faCogs\ Skills}
\begin{itemize}[parsep=0.5ex]
  \item Programming Languages: C++, C, Python, Go, HTML, CSS, javascript, Java, SQL, \LaTeX
  \item Tools \& Framework: Git, Bash, PyTorch, React, Node.js, jQuery
  %\item Other: Fluent in Chinese and English
\end{itemize}

% \datedsubsection{\textbf{Movie Archive Tracker }  \small{C++} }{Sept. 2018 -- Oct. 2018}
% \begin{itemize}
% \item Implemented a thread library utilizing linux system infrastructures supporting multiple simulated CPUs
% \item Provided full multithreading support for library interfaces such as thread.join() and mutex/condition queue
% \item Implemented a disk scheduler program that reads a series of input representing the track number of disk requests that will be executed synchronously by a disk servicer in SSTF order
% \end{itemize}

%% Reference
%\newpage
%\bibliographystyle{IEEETran}
%\bibliography{mycite}
\end{document}
