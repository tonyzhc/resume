\documentclass{resume}
%\usepackage{zh_CN-Adobefonts_external} % Simplified Chinese Support using external fonts (./fonts/zh_CN-Adobe/)
%\usepackage{zh_CN-Adobefonts_internal} % Simplified Chinese Support using system fonts

%Shout out to https://github.com/billryan/resume for providing this amazing resume template!

\begin{document}
\pagenumbering{gobble} % suppress displaying page number

\name{Hanchi Zhang}

\basicInfo{
  	\email{tonyzhc@umich.ed} \textperiodcentered\ 
 	\phone{(+1) 734-882-7007} \textperiodcentered\ 
 	\linkedin[tonyhanchizhang]{https://www.linkedin.com/in/tonyhanchizhang} \textperiodcentered\
 	\github[tonyzhc]{https://github.com/tonyzhc}} 

\section{\faGraduationCap\ Education}
\datedsubsection{\textbf{University of Michigan Ann Arbor}, Ann Arbor, MI}{Sept. 2017 - May 2020}
\textit{B.S.} in Computer Science / Data Science \hfill GPA 3.84/4.00 
%Relevant Courses: Intro to Operating System / Intro to Computer Security / Database System Management

\section{\faUsers\ Experience}
\datedsubsection{\textbf{University of Michigan Database Research Group} Ann Arbor, MI}{Apr. 2019 - Present}
\role{Research Assistant}{Python / pytorch}
\vspace{-1ex}
\begin{itemize}
	\item Implemented a video database that improves querying time of the user given model by \%.
	\item Trained multiple binary CNN classifiers 
\end{itemize}

\datedsubsection{\textbf{Perch Research} Ann Arbor, MI}{Dec. 2018 -- Present}
\role{Frontend Developer}{HTML / CSS / javascript / React}
\vspace{-1ex}
\begin{itemize}
	\item Developed frontend components such as an editable profile tab for a website using libraries including emotion.sh and styled for inline css styling improving file structure of the project
	\item Utilized React Hooks API to restructure componenets declared as classes to functions with useState / useReducer to eliminate unnecessary code structure achieve better code readability and reusability
\end{itemize}

\datedsubsection{\textbf{Beijing Xiaomi Technology}}{Jun. 2018 -- Aug. 2018}
\role{Summer Software Development Intern}{Python / Java}
\vspace{-1ex}
\begin{itemize}
	\item Developed an Android app demo providing intelligent sketch of MIUI users' daily behaviors e.g. app usage
	\item Set up a simulated server to communicate dummy data with the app using Python and flask
  	\item Optimized the hyperparameters of a deep neural network of the ad system and increased the AUC by 1\%
 	\item Implemented more complex neural network structures such as RNN/W\&D and tested their performance
\end{itemize}

% Reference Test
%\datedsubsection{\textbf{Paper Title\cite{zaharia2012resilient}}}{May. 2015}
%An xxx optimized for xxx\cite{verma2015large}
%\begin{itemize}
%  \item main contribution
%\end{itemize}

\section{\faCogs\ Skills}
\begin{itemize}[parsep=0.5ex]
  \item Programming Languages: C++, C, Python, Java, HTML, CSS,  javascript, SQL, \LaTeX, Swift (Limited)
  \item Tools: Git, Bash, Tensorflow, React, Node.js, jQuery
  \item Other: Fluent in Chinese and English
\end{itemize}

\section{\faWrench\ Projects}
\datedsubsection{\textbf{U of M Blue Bus Time Checker } \small{Python / pyQuery / AWS} }{Feb. 2019}
A Makeathon event project sponsored by Amazon co-developed under 24 hours
\begin{itemize}
	\item Developed a fully-deployable / demo-able Alexa skill that informs user the time of the next bus the user can catch and the time to walk to the closest bus station utilizing Google Matrix APIs
	\item Utilized a python library \href{https://github.com/aws/chalice}{\underline{Chalice}} to develop a backend API that can be accessed in the frontend python script and can be easily deployed to AWS Lambda
	\item Provided data by writing a simple crawler that utilizes requests / pyQuery / BS4 to access the bus time
\end{itemize}
\textit{Our project was published as a \href{https://medium.com/voice-tech-podcast/alexa-when-is-the-next-bus-620e1aba9474}{\underline{Medium blog}} by a participating Amazon Solution Architect \href{https://www.linkedin.com/in/jpd44/}{\underline{John Dixon}}}

\datedsubsection{\textbf{Multithreaded Library / Disk Scheduler }  \small{C++} }{Jan. 2019 -- Feb. 2019}
\begin{itemize}
	\item Implemented a thread library utilizing linux system infrastructures including ucontext.h that supports multithreading across multiple simulated CPU cores
	\item Provided full multithreading support for library interfaces such as and mutex/condition queue
	\item Implemented a disk scheduler program that reads a series of input representing the track number of disk requests that will be executed synchronously by a disk servicer
\end{itemize}

\datedsubsection{\textbf{Movie Archive Tracker }  \small{C++} }{Sept. 2018 -- Oct. 2018}
\begin{itemize}
\item Implemented a thread library utilizing linux system infrastructures supporting multiple simulated CPUs
\item Provided full multithreading support for library interfaces such as thread.join() and mutex/condition queue
\item Implemented a disk scheduler program that reads a series of input representing the track number of disk requests that will be executed synchronously by a disk servicer in SSTF order
\end{itemize}

%% Reference
%\newpage
%\bibliographystyle{IEEETran}
%\bibliography{mycite}
\end{document}
